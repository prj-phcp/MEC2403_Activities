\documentclass[10pt, a4paper]{article}
% \usepackage[english]{babel}
\usepackage[brazilian]{babel}
\usepackage[utf8]{inputenc}
% \usepackage[T1]{fontenc}
\usepackage{lipsum}

% code
\usepackage{pythonhighlight}
\renewcommand{\lstlistingname}{Anexo} % Listing->Code
\usepackage{adjustbox}

% For subfigure use
\usepackage[font=small,labelfont=bf]{caption}
\usepackage{subcaption}

% Set page size and margins
% Replace `letterpaper' with`a4paper' for UK/EU standard size
\usepackage[a4paper,top=2cm,bottom=2cm,left=2cm,right=2cm,marginparwidth=2cm]{geometry}

% tabelas
\usepackage{array}
\usepackage{tabularx}
\usepackage{booktabs}

\usepackage{float}

% Useful packages
\usepackage{amsmath}
\usepackage{enumerate}

\usepackage{graphicx}
\usepackage[colorlinks=true, allcolors=blue]{hyperref}
\usepackage{cleveref}
%\usepackage[notransparent]{svg}
\newcommand{\crefrangeconjunction}{--}


\begin{document}

\def\TITLE{Lista 01}
\def\DISCIPLINE{MEC 2403 - Otimização e Algoritmos para Engenhria Mecânica}
\def\PROFESSOR{Ivan Menezes}
\def\AUTHOR{Pedro Henrique Cardoso Paulo}
\def\CONTACT{pedrorjpaulo.phcp@gmail.com}
\def\DATE{abril de 2023}

\title{\textbf{\TITLE} \\ \DISCIPLINE}
\author{\AUTHOR}
\date{\DATE}

\begin{titlepage}
      \begin{center}
          \vspace*{1cm}

          \Huge
          \textbf{\TITLE}

          \vspace{0.5cm}
          \LARGE
          \DISCIPLINE

          \vspace{1.5cm}

          \textbf{\AUTHOR \\ {\tt \CONTACT}}

          \vfill
          Professor: \PROFESSOR

          \vspace{0.8cm}

          \includegraphics[width=0.2\textwidth]{../general/puc.jpg}

          \Large
          Departamento de Engenharia Mecânica\\
          PUC-RJ Pontifícia Universidade Católica do Rio de Janeiro\\
          \DATE

      \end{center}
  \end{titlepage}

\maketitle

\section{Introdução}

\subsection{Objetivos}

Esse é o entregável da \TITLE \ da disciplina \DISCIPLINE. Esse trabalho tem como objetivos:

\begin{enumerate}
  \item Implementar os principais métodos para cálculo de ponto de mínimo em funções de uma variável
  \item Aplicar esses métodos em funções 2D ao longo de uma dada direção
  \item Exercitar a linguagem de programação e as ferramentas de visualização gráfica
\end{enumerate}

\subsection{Links úteis}\label{links}

Nesta seção são listados alguns links e referências úteis para se entender o trabalho desempenhado.

\begin{enumerate}
  \item \href{https://web.tecgraf.puc-rio.br/~ivan/MEC2403/ProgMatematica_VazPereiraMenezes-Ago2012.pdf}{Apostila de programação matemática da disciplina}
  \item \href{https://github.com/prj-phcp/MEC2403_Activities}{GitHub usado para essa disciplina}
  \item \href{https://github.com/prj-phcp/MEC2403_Activities/blob/master/Lista1/Lista1.ipynb}{Notebook com o código para as figuras desse relatório}
  \item \href{https://github.com/prj-phcp/MEC2403_Activities/blob/master/packages}{Pasta com os códigos a serem aproveitados em todas as listas}
\end{enumerate}

\section{Questão 01}\label{sec:q01}

\subsection{Enunciado}

Implementar, usando o MATLAB ou Python, os seguintes métodos para cálculo do ponto
de mínimo de funções de uma única variável:

\begin{itemize}
  \item Passo constante (com $\Delta\alpha = 0.01$)
  \item Bisseção (usando o Passo Constante para obtenção do intervalo de busca)
  \item Seção Áurea (usando o Passo Constante para obtenção do intervalo de busca)
\end{itemize}

\subsection{Solução}

De modo a garantir o reaproveitamento do código para tarefas futuras, os três métodos foram implementados como classes Python, garantindo a 
possibilidade de herança entre classes. Um esquemático das 3 classes implementadas é mostrado na Figura \ref{fig:q1_1}, onde nota-se que foi convencionado
ter como classe-pai das demais a classe do passo constante. Esse arranjo foi considerado adequado uma vez que o passo constante é o método básico do qual todos
os demais métodos partem para refinar a estimativa inicial de passo.

\begin{figure}[htpb]
  \centering
  \includegraphics[width=0.8\textwidth]{images/classes.pdf}
  \caption{Estrutura de classes implementada e heranças}
  \label{fig:q1_1}
\end{figure}

O código completo das classes implementadas pode ser visto no arquivo adicionado ao GitHub 
{\tt \href{https://github.com/prj-phcp/MEC2403_Activities/blob/master/packages/steps.py}{steps.py}},
onde a implementação completa das 3 classes está armazenada. Abaixo, para fins de exemplo, é mostrada a implementação da classe de passo constante, da qual todas 
as demais herdam.

\begin{python}
class ConstantStep:
    
    def __init__(self, da):

        self.da = da
        self.aL = None
        self.aU = None
        self.fL = None
        self.fU = None

        self.reset_step()

    def reset_step(self):

        self.aL = 0
        self.aU = self.da
        self.fL =  0.0
        self.fU = -1.0

    def calculate_bounds(self, p_initial, direction, function):

        self.fL = function(*(p_initial + self.aL*direction))
        self.fU = function(*(p_initial + self.aU*direction))

    def __call__(self, p_initial, direction, function):
        
        self.reset_step()
        self.calculate_bounds(p_initial, direction, function)
        while self.fL > self.fU:
            self.aL = self.aU
            self.aU += self.da
            self.calculate_bounds(p_initial, direction, function)
        pend = p_initial + self.aL*direction
        return self.aL, pend
\end{python}

A usabilidade das classes de passo implementadas é feita de forma similar à de uma função graças à implementação de um método {\tt \_\_call\_\_}
em seu corpo. Dessa forma, uma vez declarado o step com seus parâmetros, uma quantidade infinita de passos podem ser dados fornecendo-se como
entrada a função, ponto inicial e direção. Um exemplo do uso dessa classe pode ser visto abaixo.

\begin{python}
import numpy as np
import steps

# Funcao a ser otimizada
def f(x1, x2):
    return np.sin(x1 + x2) + (x1 - x2)**2 - 1.5*x1 + 2.5*x2

# Ponto inicial e direcao
p_inicial = np.array([-2, 3])
d = np.array([1.453, -4.547])

# Instanciando o passo
stp = steps.ConstantStep(da = 0.01)

# Dando o passo
ak, p_final = stp(p_inicial, d, f)
\end{python}

\newpage
\section{Questão 02}

\subsection{Enunciado}

Utilizando os métodos implementados na questão anterior, testar a sua
implementação encontrando o ponto de mínimo das seguintes funções:

\begin{enumerate}[(a)]
  \item Primeiro Exemplo: \\
        $f(x_1, x_2) = x_1^2 - 3x_1x_2 + 4x_2^2 + x_1 - x_2$ \\
        Ponto inicial: $\mathbf{x^0} = [1, 2]^T$, Direção: $\mathbf{d} = [-1, -2]^T$\label{func:a}
  \item Função de McCormick: \\
        $f(x_1, x_2) = \sin{(x_1 + x_2)} + (x_1 - x_2)^2 - 1.5x_1 + 2.5x_2$ \\
        Ponto inicial: $\mathbf{x^0} = [-2, 3]^T$, Direção: $\mathbf{d} = [1.453, -4.547]^T$\label{func:b}
  \item Função de Himmelblau: \\
        $f(x_1, x_2) = (x_1^2 + x_2 - 11)^2 + (x_1 + x_2^2 - 7)^2$ \\
        Ponto inicial: $\mathbf{x^0} = [0, 5]^T$, Direção: $\mathbf{d} = [3, 1.5]^T$\label{func:c}
\end{enumerate}

Para cada função acima, utilize o MATLAB ou Python para desenhar (na mesma figura) as
curvas de nível e o segmento de reta conectando o ponto inicial ao ponto de mínimo.
Adotar tolerância de $10^{-5}$ para verificação da convergência numérica.

\subsection{Solução}

Para executar esse exercício, serão usadas as classes já descritas na Questão 01 (seção \ref{sec:q01}). A biblioteca {\tt matplotlib} do Python será 
novamente usada para gerar os gráficos necessários. Para facilitar a vaisualização, será gerado um gráfico para cada método de cálculo do passo, sendo que 
cada gráfico deverá conter, pelo menos:

\begin{itemize}
  \item As curvas de nível da função a ser estudada
  \item Os pontos iniciais e finais do passo
  \item A direção esperada do passo
  \item Uma seta ou linha ligando os pontos iniciais e finais
  \item O tamanho do passo $\alpha_k$ e coordenadas do ponto final (incluindo $z = f(x_1, x_2)$) no título do gráfico
\end{itemize}

Um exemplo simples do código que foi usado para gerar os gráficos apresentados neste relatório é apresentado abaixo. 
Em seguida, são apresentadas as seções que mostram os resultados obtidos e alguns comentários a respeito destes.


\begin{python}
step_list   = [steps.ConstantStep(da = 0.01), steps.BissectionStep(da = 0.01, tol = 1e-5), steps.GoldenSectionStep(da = 0.01, tol = 1e-5)]

item = 'a'
x = np.linspace(-3.3, 0.1, 50)
y = np.linspace(3, 5.1, 50)
X, Y = np.meshgrid(x, y)
Z = f(X, Y)

i = 0
for step in step_list:
    i += 1
    fig, ax = plt.subplots(1,1, figsize=(5, 5))
    ax.contour(X, Y, Z, cmap='rainbow')
    ax.plot(*p_inicial, 'ro')
    ax.text(p_inicial[0]-0.25, p_inicial[1]+0.04, '$P_{inic}$', color='red')
    ax.plot(*p_final, 'ro')
    ax.text(p_final[0]-0.25, p_final[1]+0.04, '$P_{final}$', color='red')
    ax.quiver(p_inicial[0], p_inicial[1], p_final[0]-p_inicial[0], p_final[1]-p_inicial[1], color='red', angles='xy', scale_units='xy', scale=1)
    ax.quiver(p_inicial[0], p_inicial[1], d[0], d[1], color='black', angles='xy', label='Direcao do passo')
    ax.grid()
    ax.legend()
    ax.set_xlabel('$x_1$')
    ax.set_ylabel('$x_2$')
    ax.set_title(f'$\\alpha_k = {ak:.5f}$, $\mathbf{{P_{{final}}}} = [{p_final[0]:.2f}, {p_final[1]:.2f}]^T$')
\end{python}
\subsubsection{Resultados para a função (\ref{func:a})}

Os resultados para a análise da função (\ref{func:a}) são apresentados na Figura \ref{fig:q2a}. Analisando os resultados
é possível ver que os três méetodos retornaram resultados bem similares, com os étodos da Bisseção e Seção áurea retornando
uma estimativa ligeiramente mais refinada para o mínimo, com mínimas diferenças no $\alpha_k$ estimado. É interessante notar que,
para este exemplo, o ponto estimado parece ser aproximadamente o ponto em que a direção em que tentamos calcular o mínimo
tangencia uma curva de nível, o que convergeria para um ponto em que esperaríamos ter o primeiro mínimo local da função nessa direção.

% Figura da funcao a
\begin{figure}[htpb]
  \centering
  \begin{subfigure}[b]{0.32\textwidth}
      \centering
      \includegraphics[width=\textwidth]{images/q2a_1.pdf}
      \caption{Passo constante}
      \label{fig:q2a_1}
  \end{subfigure}
  \hfill
  \begin{subfigure}[b]{0.32\textwidth}
      \centering
      \includegraphics[width=\textwidth]{images/q2a_2.pdf}
      \caption{Bisseção}
      \label{fig:q2a_2}
  \end{subfigure}
  \hfill
  \begin{subfigure}[b]{0.32\textwidth}
      \centering
      \includegraphics[width=\textwidth]{images/q2a_3.pdf}
      \caption{Seção áurea}
      \label{fig:q2a_3}
  \end{subfigure}
  \caption{Resultados para a função (\ref{func:a}) caso com diferentes métodos}
  \label{fig:q2a}
\end{figure}

\subsubsection{Resultados para a função (\ref{func:b})}

Os resultados para a análise da função (\ref{func:b}) são apresentados na Figura \ref{fig:q2b}. Novamente, não há 
diferenças apreciáveis entre os resultados dos métodos. Isso provavelmente se deve ao fato de o $\Delta\alpha$ especificado para
o passo constante (e usado de base para delinear o intervalo dos demais métodos) já ser refinado o suficiente para 
obtermos uma resposta adequada sem a necessidade de uma posterior Bisseção ou Seção Áurea.

% Figura da funcao b
\begin{figure}[htpb]
  \centering
  \begin{subfigure}[b]{0.32\textwidth}
      \centering
      \includegraphics[width=\textwidth]{images/q2b_1.pdf}
      \caption{Passo constante}
      \label{fig:q2b_1}
  \end{subfigure}
  \hfill
  \begin{subfigure}[b]{0.32\textwidth}
      \centering
      \includegraphics[width=\textwidth]{images/q2b_2.pdf}
      \caption{Bisseção}
      \label{fig:q2b_2}
  \end{subfigure}
  \hfill
  \begin{subfigure}[b]{0.32\textwidth}
      \centering
      \includegraphics[width=\textwidth]{images/q2b_3.pdf}
      \caption{Seção áurea}
      \label{fig:q2b_3}
  \end{subfigure}
     \caption{Resultados para a função (\ref{func:b}) com diferentes métodos}
     \label{fig:q2b}
\end{figure}

\subsubsection{Resultados para a função (\ref{func:c})}

Os resultados para a análise da função (\ref{func:c}) são apresentados na Figura \ref{fig:q2c}. Diferentemente dos casos anteriores, aqui vemos
que os três métodos convergiram pra pontos iguais ou muito próximos do ponto inicial, ou seja, nosso ponto inicial já é o 
mínimo esperado. Isso é algo que faz sentido ao notarmos que a direção em que buscamos o nosso mínimo tem sentido apontando
para curvas de nível de maior valor, ou seja, sem uma proteção para esse caso, a busca necessariamente leva a valores maiores da função na proximidade
do ponto inicial.

% Figura da funcao c
\begin{figure}[htpb]
  \centering
  \begin{subfigure}[b]{0.32\textwidth}
      \centering
      \includegraphics[width=\textwidth]{images/q2c_1.pdf}
      \caption{Passo constante}
      \label{fig:q2c_1}
  \end{subfigure}
  \hfill
  \begin{subfigure}[b]{0.32\textwidth}
      \centering
      \includegraphics[width=\textwidth]{images/q2c_2.pdf}
      \caption{Bisseção}
      \label{fig:q2c_2}
  \end{subfigure}
  \hfill
  \begin{subfigure}[b]{0.32\textwidth}
      \centering
      \includegraphics[width=\textwidth]{images/q2c_3.pdf}
      \caption{Seção áurea}
      \label{fig:q2c_3}
  \end{subfigure}
     \caption{Resultados para a função (\ref{func:c}) com diferentes métodos}
     \label{fig:q2c}
\end{figure}

Para acharmos o mínimo para a direção dada, seria necessário efetuarmos a busca no sentido contrário ao do vetor direção fornecido, 
o que é análogo a procurarmos valores negativos de $\alpha$. A Figura \ref{fig:q2c2} mostra o resultado que seria obtido 
fazendo-se a busca na mesma direção mas com vetor de sentido contrário. Nota-se que nesse caso o 
método seguiu até um ponto de mínimo inferior ao ponto inicial, evidenciando o argumento de que, além de termos algoritmos capazes de 
estimar corretamente o $\alpha_k$, é igualmente importante sermos capazes de estimar adequadamente o sentido a percorrer na direção de busca.

% Figura da funcao c - reversa
\begin{figure}[htpb]
  \centering
  \begin{subfigure}[b]{0.32\textwidth}
      \centering
      \includegraphics[width=\textwidth]{images/q2c2_1.pdf}
      \caption{Passo constante}
      \label{fig:q2c2_1}
  \end{subfigure}
  \hfill
  \begin{subfigure}[b]{0.32\textwidth}
      \centering
      \includegraphics[width=\textwidth]{images/q2c2_2.pdf}
      \caption{Bisseção}
      \label{fig:q2c2_2}
  \end{subfigure}
  \hfill
  \begin{subfigure}[b]{0.32\textwidth}
      \centering
      \includegraphics[width=\textwidth]{images/q2c2_3.pdf}
      \caption{Seção áurea}
      \label{fig:q2c2_3}
  \end{subfigure}
     \caption{Resultados para a função (\ref{func:c}) com diferentes métodos (revertendo a direção)}
     \label{fig:q2c2}
\end{figure}

De modo a robustecer a implementação para usos futuros, a Figura \ref{fig:q2c2} foi gerada por meio
da implementação de uma checagem de derivada numérica idêntica a feita no método da Bisseção para 
selecionar a metade a manter no início do método do passo constante. Essa checagem, por sua vez, definiu
um multiplicador a ser aplicado em $\mathbf{d}$ durante o processo e reportado em $\alpha$ como 
seu sinal. Para fins de exemplo, um valor booleano adicional foi posto na declaração da função passo
para desligar essa checagem, de modo a permitir testes em que seguiríamos cegamente a o sentido da direção dada,
como o que gerou a Figura \ref{fig:q2c}.

%%%%%%%%%%%%%%%%%%%%%%%%%%%%%%%%%%%%%%%%%%%%%%%%%%%

\bibliographystyle{apalike}
\bibliography{export}

\end{document}
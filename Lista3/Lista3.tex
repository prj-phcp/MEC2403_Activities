\documentclass[10pt, a4paper]{article}
% \usepackage[english]{babel}
\usepackage[brazilian]{babel}
\usepackage[utf8]{inputenc}
% \usepackage[T1]{fontenc}
\usepackage{lipsum}

% code
\usepackage{pythonhighlight}
\renewcommand{\lstlistingname}{Anexo} % Listing->Code
\usepackage{adjustbox}

% For subfigure use
\usepackage[font=small,labelfont=bf]{caption}
\usepackage{subcaption}

% Set page size and margins
% Replace `letterpaper' with`a4paper' for UK/EU standard size
\usepackage[a4paper,top=2cm,bottom=2cm,left=2cm,right=2cm,marginparwidth=2cm]{geometry}

% tabelas
\usepackage{array}
\usepackage{tabularx}
\usepackage{booktabs}

\usepackage{float}
\usepackage{pdfpages} %Insert pdf pages

% Useful packages
\usepackage{amsmath}
\usepackage{enumerate}

\usepackage{graphicx}
\usepackage[colorlinks=true, allcolors=blue]{hyperref}
\usepackage{cleveref}
%\usepackage[notransparent]{svg}
\newcommand{\crefrangeconjunction}{--}


\begin{document}

\def\TITLE{Lista 03}
\def\DISCIPLINE{MEC 2403 - Otimização e Algoritmos para Engenharia Mecânica}
\def\PROFESSOR{Ivan Menezes}
\def\AUTHOR{Pedro Henrique Cardoso Paulo}
\def\CONTACT{pedrorjpaulo.phcp@gmail.com}
\def\DATE{junho de 2023}

\title{\textbf{\TITLE} \\ \DISCIPLINE}
\author{\AUTHOR}
\date{\DATE}

\begin{titlepage}
      \begin{center}
          \vspace*{1cm}

          \Huge
          \textbf{\TITLE}

          \vspace{0.5cm}
          \LARGE
          \DISCIPLINE

          \vspace{1.5cm}

          \textbf{\AUTHOR \\ {\tt \CONTACT}}

          \vfill
          Professor: \PROFESSOR

          \vspace{0.8cm}

          \includegraphics[width=0.2\textwidth]{../general/puc.jpg}

          \Large
          Departamento de Engenharia Mecânica\\
          PUC-RJ Pontifícia Universidade Católica do Rio de Janeiro\\
          \DATE

      \end{center}
  \end{titlepage}

%\maketitle

\includepdf[pages=-]{Lista3_editada.pdf}

\appendix

\appendixname

\section{Gráficos iterativos}

Os gráficos forma gerados usando a calculadora online Desmos. Caso seja do interesse ver e interagir, visitar:

\begin{itemize}
  \item \textbf{Questão 2}: \href{https://www.desmos.com/calculator/nu2funszid}{Clique aqui}
  \item \textbf{Questão 3}: \href{https://www.desmos.com/calculator/weifmlruuk}{Clique aqui}
  \item \textbf{Questão 4}: \href{https://www.desmos.com/calculator/a0nfpi3fnx}{Clique aqui}
  \item \textbf{Questão 5}: \href{https://www.desmos.com/calculator/q8xrue3lws}{Clique aqui}
\end{itemize}

%%%%%%%%%%%%%%%%%%%%%%%%%%%%%%%%%%%%%%%%%%%%%%%%%%%

\bibliographystyle{apalike}
\bibliography{export}

\end{document}
\documentclass[10pt, a4paper]{article}
% \usepackage[english]{babel}
\usepackage[brazilian]{babel}
\usepackage[utf8]{inputenc}
% \usepackage[T1]{fontenc}
\usepackage{lipsum}

% code
\usepackage{pythonhighlight}
\renewcommand{\lstlistingname}{Anexo} % Listing->Code
\usepackage{adjustbox}

% For subfigure use
\usepackage[font=small,labelfont=bf]{caption}
\usepackage{subcaption}

% Set page size and margins
% Replace `letterpaper' with`a4paper' for UK/EU standard size
\usepackage[a4paper,top=2cm,bottom=2cm,left=2cm,right=2cm,marginparwidth=2cm]{geometry}

% tabelas
\usepackage{array}
\usepackage{tabularx}
\usepackage{booktabs}

\usepackage{float}

% Useful packages
\usepackage{amsmath}
\usepackage{enumerate}

\usepackage{graphicx}
\usepackage[colorlinks=true, allcolors=blue]{hyperref}
\usepackage{cleveref}
%\usepackage[notransparent]{svg}
\newcommand{\crefrangeconjunction}{--}


\begin{document}

\def\TITLE{Lista 02}
\def\DISCIPLINE{MEC 2403 - Otimização e Algoritmos para Engenharia Mecânica}
\def\PROFESSOR{Ivan Menezes}
\def\AUTHOR{Pedro Henrique Cardoso Paulo}
\def\CONTACT{pedrorjpaulo.phcp@gmail.com}
\def\DATE{maio de 2023}

\title{\textbf{\TITLE} \\ \DISCIPLINE}
\author{\AUTHOR}
\date{\DATE}

\begin{titlepage}
      \begin{center}
          \vspace*{1cm}

          \Huge
          \textbf{\TITLE}

          \vspace{0.5cm}
          \LARGE
          \DISCIPLINE

          \vspace{1.5cm}

          \textbf{\AUTHOR \\ {\tt \CONTACT}}

          \vfill
          Professor: \PROFESSOR

          \vspace{0.8cm}

          \includegraphics[width=0.2\textwidth]{../general/puc.jpg}

          \Large
          Departamento de Engenharia Mecânica\\
          PUC-RJ Pontifícia Universidade Católica do Rio de Janeiro\\
          \DATE

      \end{center}
  \end{titlepage}

\maketitle

\section{Introdução}

\subsection{Objetivos}

Esse é o entregável da \TITLE \ da disciplina \DISCIPLINE. Esse trabalho tem como objetivos:

\begin{enumerate}
  \item Implementar os principais métodos determinação de direções para otimização em múltiplas variáveis
  \item Aplicar esses métodos em funções 2D com o uso de passos analíticos
  \item Exercitar a linguagem de programação e as ferramentas de visualização gráfica
\end{enumerate}

\subsection{Links úteis}\label{links}

Nesta seção são listados alguns links e referências úteis para se entender o trabalho desempenhado.

\begin{enumerate}
  \item \href{https://web.tecgraf.puc-rio.br/~ivan/MEC2403/ProgMatematica_VazPereiraMenezes-Ago2012.pdf}{Apostila de programação matemática da disciplina}
  \item \href{https://github.com/prj-phcp/MEC2403_Activities}{GitHub usado para essa disciplina}
  \item \href{https://github.com/prj-phcp/MEC2403_Activities/blob/master/Lista2/Lista2.ipynb}{Notebook com o código para as figuras desse relatório}
  \item \href{https://github.com/prj-phcp/MEC2403_Activities/blob/master/packages}{Pasta com os códigos a serem aproveitados em todas as listas}
\end{enumerate}

\section{Questão 01}\label{sec:q01}

\subsection{Enunciado}

Dada a função $f(x_1, x_2) = x_1^2 - 3x_1x_2 + 4x_2^2 + x_1 - x_2$, minimizá-la partindo do ponto $\mathbf{x^0} = [2, 2]^T$ utilizando os métodos de otimização: 
(a) \textit{Univariante}; (b) \textit{Powell}; (c) \textit{Steepest Descent}; 
(d) \textit{Fletcher–Reeves}; (e) \textit{BFGS}; e (f) \textit{Newton–Raphson}.

Preencher uma tabela com os resultados obtidos adotando uma tolerência de $10^-5$ e um número máximo de 3 passos para cada método. Para cada passo (iteração)
de cada método indicar o valor de $\alpha$ obtido na busca linear.


\subsection{Solução}

E execução deste exercício demandou uma pequena refatoração do código do pacote {\tt \href{https://github.com/prj-phcp/MEC2403_Activities/blob/master/packages/steps.py}{steps.py}},
desenvolvido para a solução da \href{https://github.com/prj-phcp/MEC2403_Activities/blob/master/Lista1/Lista1.pdf}{Lista 01}. Essa refatoração implicou na criação de uma 
classe de passo genérico ({\tt GenericStep}) da qual passos dissociados do passo constante pudessem herdar. Além disso, mais dois arquivos foram criados: 
{\tt \href{https://github.com/prj-phcp/MEC2403_Activities/blob/master/packages/functions.py}{functions.py}}, onde foi definido um objeto capaz de aramzenar as informações
de uma função, seu gradiente e sua Hessiana para casos analíticos e numéricos e {\tt \href{https://github.com/prj-phcp/MEC2403_Activities/blob/master/packages/optimizers.py}{optimizers.py}},
onde foram definidos objetos responsáveis pela execução dos métodos de otimização (sua definição de direção de busca e iteração até a convergência). A Figura \ref{fig:q1_1}
exemplifica o resultado final das classes implementadas. Mais detalhes da implementação dos otimizadores será apresentado no Trabalho 01, em elaboração.

\begin{figure}[htpb]
  \centering
  \includegraphics[width=0.8\textwidth]{../general/classes_full.pdf}
  \caption{Estrutura de classes implementada e heranças}
  \label{fig:q1_1}
\end{figure}

Como o principal objetivo do presente exercício é validar a implementação dos métodos de otimização e a função a ser estudada é sabidamente quadrática, dada uma direção de busca $\mathbf{d}_i$,
o valor do passo linear ideal $\alpha_i$ pode ser calculado com base no gradiente da função $\mathbf{\nabla}f$ e em sua Hessiana $\mathbf{H}(f)$ por meio da equação
\ref{eq:q1_1}.

\begin{equation}\label{eq:q1_1}
  \alpha_i = \frac{\mathbf{\nabla}f^T \mathbf{d}_i}{\mathbf{d}_i^T \mathbf{H}(f) \mathbf{d}_i}
\end{equation}

Para a função $f(x_1, x_2)$ descrita no enunciado, o gradiente e sua Hessiana são descritos nas equações \ref{eq:q1_2} e \ref{eq:q1_3}.

\begin{equation}\label{eq:q1_2}
  \mathbf{\nabla}f = \begin{bmatrix} 2x_1 - 3x_2 + 1\\-3x_1 + 8x_2 - 1 \end{bmatrix}
\end{equation}

\begin{equation}\label{eq:q1_3}
  \mathbf{H}(f) = \begin{bmatrix} 2 &-3 \\ -3 & 8 \end{bmatrix}
\end{equation}

Essa equação foi implementada como um objeto chamado {\tt AnalyticalStep} para ser passada aos métodos de otimização. O código desse objeto é mostrado abaixo:

\begin{python}
class AnalyticalStep(GenericStep):

  def __init__(self):

    super().__init__()

  def __call__(self, p_initial, direction, function):
    

    grad = function.grad(*p_initial).reshape(-1,1)
    direction = direction.reshape(-1,1)
    Q = function.Hessian(*p_initial)

    ak = - np.dot(grad.T, direction) / (direction.T @ Q @ direction)
    ak =  ak.reshape(-1)[0]

    pend = pend = p_initial + ak*direction.reshape(-1)

    return ak, pend
\end{python}

Ressalta-se que o critério de parada usado para os otimizadores foi o critério de parada aplicado para a otimização foi, no presente estudo, um limite de 3 iterações
e uma condção de chegada ao ponto crítico definida pela equação \ref{eq:q1_4}, com a tolerência $tol$ sendo igual a $10^{-5}$.

\begin{equation}\label{eq:q1_4}
  \left|\mathbf{\nabla}f\right| \leq tol
\end{equation}

Os resultados finais das 3 iterações executadas são resumidos na Tabela \ref{tab:results_summ}, com os resultados por passo em forma gráfica sendo mostrados na 
Figura \ref{fig:q1_2}. Nota-se a partir dos resultados que os valores finais obtidos foram coerentes com o previsto pela teoria, com os métodos Fletcher-Reeves,
Newton-Raphson e BFGS sendo os únicos a terem convergido no total de 3 passos ou menos, conforme esperado. Além disso, nota-se que o passo analítico para o método
de Newton-Raphson retornou também um valor de 1 na única iteração que este precisou, o que novamente está de acordo com o esperado pelo desenvolvimento teórico
do método. Também é interessante notar que os métodos de Powell e Univariante apresentaram passos iguais nas duas primeiras iterações, algo que novamente corrobora
com a boa implementação dos métodos dado que eles são idênticos nessas duas iterações.

\begin{table}[htpb]
  \centering
  \begin{tabular}{l|c|c|c|c|c|}
    Método             &	Ponto de mínimo	                     & Passos	 & $\alpha_1$   & $\alpha_2$	& $\alpha_3$ \\
    \hline
    Univariante        & $[ 1.093750,  1.062500,  2.256836]^T$ & 3       &  0.50000     & -0.93750    & -1.40625      \\
    Powell             & $[ 2.432024,  1.189955,  4.138784]^T$ & 3       &  0.50000     & -0.93750    & -0.13596      \\
    Steepest Descent   & $[ 0.484934,  0.320841,  0.344249]^T$ & 3       &  0.11647     &  0.70690    &  0.11648      \\
    Fletcher-Reeves    & $[-0.714286, -0.142857, -0.285714]^T$ & 2       &  0.11647     &  1.22648    &  --           \\
    Newton-Raphson     & $[-0.714286, -0.142857, -0.285714]^T$ & 1       &  1.00000     &  --         &  --           \\
    BFGS               & $[-0.714286, -0.142857, -0.285714]^T$ & 2       &  0.11647     &  1.22648    &  --           \\
    \hline
  \end{tabular}
  \caption{Resumo dos resultados obtidos}
  \label{tab:results_summ}
\end{table}


\begin{figure}[htpb]
  \centering
  \begin{subfigure}[b]{0.32\textwidth}
      \centering
      \includegraphics[width=\textwidth]{images/q1_Univariant.pdf}
      \caption{Univariante}
      \label{fig:q1_univariant}
  \end{subfigure}
  \hfill
  \begin{subfigure}[b]{0.32\textwidth}
    \centering
    \includegraphics[width=\textwidth]{images/q1_Powell.pdf}
    \caption{Powell}
    \label{fig:q1_powell}
  \end{subfigure}
  \hfill
  \begin{subfigure}[b]{0.32\textwidth}
    \centering
    \includegraphics[width=\textwidth]{images/q1_Steepest.pdf}
    \caption{Steepest Descent}
    \label{fig:q1_steepest}
  \end{subfigure}
  \hfill
  \begin{subfigure}[b]{0.32\textwidth}
    \centering
    \includegraphics[width=\textwidth]{images/q1_FletchRvs.pdf}
    \caption{Fletcher-Reeves}
    \label{fig:q1_fletchrvs}
  \end{subfigure}
  \hfill
  \begin{subfigure}[b]{0.32\textwidth}
    \centering
    \includegraphics[width=\textwidth]{images/q1_NewtnRaph.pdf}
    \caption{Newton-Raphson}
    \label{fig:q1_newtnraph}
  \end{subfigure}
  \hfill
  \begin{subfigure}[b]{0.32\textwidth}
    \centering
    \includegraphics[width=\textwidth]{images/q1_BFGS.pdf}
    \caption{BFGS}
    \label{fig:q1_bfgs}
  \end{subfigure}
     \caption{Resumo gráfico dos passos dados em cada método}
     \label{fig:q1_2}
\end{figure}

%%%%%%%%%%%%%%%%%%%%%%%%%%%%%%%%%%%%%%%%%%%%%%%%%%%

\bibliographystyle{apalike}
\bibliography{export}

\end{document}